\documentclass[12pt]{amsart}
\usepackage[symbol]{footmisc}
\usepackage{float}

\usepackage{style_template}
\renewcommand*{\thefootnote}{\fnsymbol{footnote}}

% \newtheorem{theorem}{Theorem}
% \newtheorem{definition}{Definition}

\title{Young Mathematicians Conference Notes}
\author{David Yang}
\date{August 15th - 17th, 2023}

\begin{document}

\maketitle

\section{Day 1: August 15th}

\vspace{0.25cm}

\subsection{The Failed Zero Forcing Number of a Graph}

\textit{}
\vspace{0.25cm}

\textit{Presented by Chirag Kaudan and Rachel Taylor.}

\begin{definition}[Forcing Rule]
Let each vertex of a graph represent a person. Each person either knows or does not know a secret -- if they do, their corresponding vertex is colored. \\

If all a person's friends except one friend knows the secret, then the secret is told to that friend as well.\end{definition}

\begin{definition}[Zero Forcing Number]
The zero forcing number of $G$, $Z(G)$, is the smallest cardinality of any set $S$ of vertices on which repeated applications of the forcing rule
results in all vertices joining $S$.
\end{definition}

\begin{definition}[Failed Zero Forcing Number]
The failed zero forcing number of $G$, $F(G)$, is the maximum
cardinality of any set of vertices on which repeated applications of the forcing rule will never result in all vertices joining the set.\end{definition}

\begin{result}
Using the theory of \textit{modules} (a set of vertices such that every vertex in the module has the same neighorhood exclusing vertices in the module) in zero forcing graphs and a computer algorithm, they were able to show that
there are $15$ graphs with $F(G) = 2$ and $68$ graphs with $F(G) = 3$. 
\end{result}

\subsection{Properties of Families of Graphs with Forbidden Induced Subgraphs}

\textit{}
\vspace{0.25cm}

\textit{Presented by Christian Pippin.}

\begin{definition}[Induced Subgraphs]
$H$ is an \textbf{induced subgraph} of $G$ if the vertex set of $H$ is a subset of the vertex set of $G$ and for all $(u, v) \in E^{H}$, $(u, v) \in E^G.$
\end{definition}

There is a relation between indivisibility and the lex product. \\

\begin{lemma}
If a family of graphs is closed under the lex product, the class is indivisible.
\end{lemma}

\begin{theorem}
For all $n$, $\mathrm{Forb}(P_n)$ is indivisible. \\

$\mathrm{Forb}(C_n)$ is indivisible for all $n \geq 5.$
\end{theorem}
\end{document}

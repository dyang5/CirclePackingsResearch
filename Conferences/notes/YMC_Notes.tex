\documentclass[12pt]{amsart}
\usepackage[symbol]{footmisc}
\usepackage{float}

\usepackage{style_template}
\renewcommand*{\thefootnote}{\fnsymbol{footnote}}

% \newtheorem{theorem}{Theorem}
% \newtheorem{definition}{Definition}

\title{Young Mathematicians Conference Notes}
\author{David Yang}
\date{August 15th - 17th, 2023}

\begin{document}

\maketitle

\section{Day 1: August 15th}

\vspace{0.25cm}

\subsection{The Failed Zero Forcing Number of a Graph}

\textit{}
\vspace{0.25cm}

\textit{Presented by Chirag Kaudan and Rachel Taylor.}

\begin{definition}[Forcing Rule]
Let each vertex of a graph represent a person. Each person either knows or does not know a secret -- if they do, their corresponding vertex is colored. \\

If all a person's friends except one friend knows the secret, then the secret is told to that friend as well.\end{definition}

\begin{definition}[Zero Forcing Number]
The zero forcing number of $G$, $Z(G)$, is the smallest cardinality of any set $S$ of vertices on which repeated applications of the forcing rule
results in all vertices joining $S$.
\end{definition}

\begin{definition}[Failed Zero Forcing Number]
The failed zero forcing number of $G$, $F(G)$, is the maximum
cardinality of any set of vertices on which repeated applications of the forcing rule will never result in all vertices joining the set.\end{definition}

\begin{result*}
Using the theory of \textit{modules} (a set of vertices such that every vertex in the module has the same neighorhood exclusing vertices in the module) in zero forcing graphs and a computer algorithm, they were able to show that
there are $15$ graphs with $F(G) = 2$ and $68$ graphs with $F(G) = 3$. 
\end{result*}

\newpage 

\subsection{Properties of Families of Graphs with Forbidden Induced Subgraphs}

\textit{}
\vspace{0.25cm}

\textit{Presented by Christian Pippin.}

\begin{definition}[Induced Subgraphs]
$H$ is an \textbf{induced subgraph} of $G$ if the vertex set of $H$ is a subset of the vertex set of $G$ and for all $(u, v) \in E^{H}$, $(u, v) \in E^G.$
\end{definition}

There is a relation between indivisibility and the lex product. \\

\begin{lemma*}
If a family of graphs is closed under the lex product, the class is indivisible.
\end{lemma*}

\begin{theorem*}
For all $n$, $\mathrm{Forb}(P_n)$ is indivisible. \\

$\mathrm{Forb}(C_n)$ is indivisible for all $n \geq 5.$
\end{theorem*}

\vspace{2.5cm}

\subsection{Generalized Stick Fragmentation and Benford's Law}

\textit{}
\vspace{0.25cm}

\textit{Presented by Xinyu Fang and Maxwell Sun.}

\begin{theorem*}[Benford's Law]
In base $B$, the probability of observing a value with first digit $d$ is \[ \log_B\left(\frac{d+1}{d}\right).\]

If this property holds, a dataset is said to exhibit \textbf{weak Benford behavior}; if, moreover, the logarithms of the values modulo 1 are equidistributed, then the set exhibits \textbf{strong Benford behavior}. 
\end{theorem*}

\begin{definition}[Significand and Mentissa]
The \textbf{significand} of $x$ base $B$ is $S_B(x) \in [1, B)$ such that $x = S_B(x) \cdot B^k$ for some $k$. \\

The \textbf{mantissa} is the analogue but for fractional pieces.
\end{definition}

\vspace{0.5cm}

To setup the problem, consider a stick breaking model where you begin with a stick of length $\ell$. At each step, the stick is broken into smaller segments and this process continues until some termination condition is reached. The problem the authors studied is known as a \textit{Discrete Breaking Problem with a Stopping Set}: if the stick length falls into the given stopping set, it is considered ``dead" and should not be broken further. \\

\begin{result*}
The authors give some results on when a given set of end lengths is Strong Benford (for which types of stopping sets). \\

In particular, they conjecture that if you break each stick into $k$ parts and stop at $(k-1) \cdot \frac{n}{k}$ residue classes modulo $n$ where $k \mid n$, then the results are Strong Benford.
\end{result*}

\vspace{2.5cm}

\subsection{Geometry of the Numerical and Berezin Range}

\textit{}
\vspace{0.25cm}

\textit{Presented by Edwin Xie and Caroline Norman.}

\begin{definition}[Numerical Range]
The \textbf{numerical range} $f$ of a bounded linear operator $T$ on a complex Hilbert space $H$ is defined as
\[ W(T) = \{\langle Tf, f \rangle : ||f|| = 1.\} \]

Properties of the numerical range include unitary invariance, shift and scale, and that it satisfies the Topelitz-Hausdorff property.
\end{definition}

\begin{definition}[Berezin Range]
On the Hardy-Hilbert space $H^2(\mathbb{D})$ of the open unit disk $\mathbb{D}$, the \textbf{Berezin range} is defined as 
\[B(T) = \{ \langle T\hat{k}_p, \hat{k}_p \rangle : p \in \mathbb{D}\}\]
where $\hat{k}_p$ is the normalized reproducing kernel.

\end{definition}
\begin{result*}
The authors work with the Berezin Range on a Hardy Space and prove some results about the Berezin Range.
\end{result*}
\end{document}
